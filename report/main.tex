% RUG LaTeX Template %

\documentclass{article}

%%%%%%%%%%%%%%%%%%%%%%%%%%%%%%%%%%%%%%%%%%%%%%%%%%%%%%%%%%%%%%%%%%%%
%%-----------------------PAGE SETTINGS----------------------------%%
%%%%%%%%%%%%%%%%%%%%%%%%%%%%%%%%%%%%%%%%%%%%%%%%%%%%%%%%%%%%%%%%%%%%
\usepackage[utf8]{inputenc}
\usepackage[margin=0.01cm]{geometry}

%%%%%%%%%%%%%%%%%%%%%%%%%%%%%%%%%%%%%%%%%%%%%%%%%%%%%%%%%%%%%%%%%%%%
%%--------------------------PREAMBLE------------------------------%%
%%%%%%%%%%%%%%%%%%%%%%%%%%%%%%%%%%%%%%%%%%%%%%%%%%%%%%%%%%%%%%%%%%%%
\usepackage{pdfpages}                           % Insert PDF into file without breaking margins and print output.
\usepackage{anyfontsize}                        % Use any font size.
\usepackage{setspace}                           % Customize paragraph spacing.
\usepackage{mathtools}                          % Math typing.
\usepackage{cancel}                             % Strikethrough text.
\usepackage{float}                              % Force figure on position.
\usepackage[hidelinks]{hyperref}	            % Links.
\usepackage{amsmath}                            % Math typing.
\usepackage{inputenc}                           % Special characters.
\usepackage{amsfonts}                           % Fontpack.
\usepackage{graphicx}                           % Graphics.
\usepackage{enumitem}                           % Special numerisation on lists.
\usepackage{amsthm}                             % Math.
\usepackage{xcolor}                             % Colors.
\usepackage{lipsum}                             % Dummy text.
\usepackage{url}								% Url's.
\usepackage{lmodern}							% Latin Modern for Computer Fonts.
\usepackage{textcomp}							% Special symbols like degrees or euro.
\usepackage{listings}							% Input code directly from file or path.
\usepackage{pdfpages}							% Include pdf documents in the output file.
\usepackage{amssymb}							% Math.
\usepackage{tikz}								% Drawing plots.

%%%%%%%%%%%%%%%%%%%%%%%%%%%%%%%%%%%%%%%%%%%%%%%%%%%%%%%%%%%%%%%%%%%%
%%-----------------------CUSTOM COMMANDS--------------------------%%
%%%%%%%%%%%%%%%%%%%%%%%%%%%%%%%%%%%%%%%%%%%%%%%%%%%%%%%%%%%%%%%%%%%%
\newcommand{\textBF}[1]{%
    \pdfliteral direct {2 Tr 1 w}               %the second factor is the boldness
     #1%
    \pdfliteral direct {0 Tr 0 w}               %
}

\newcommand{\textDF}[1]{%
    \pdfliteral direct {2 Tr 0.2 w}             %the second factor is the boldness
     #1%
    \pdfliteral direct {0 Tr 0 w}               %
}

\definecolor{codegreen}{rgb}{0,0.6,0}
\definecolor{codegray}{rgb}{0.5,0.5,0.5}
\definecolor{codepurple}{rgb}{0.58,0,0.82}
\definecolor{backcolour}{rgb}{0.95,0.95,0.92}

\lstdefinestyle{mystyle}{
    backgroundcolor=\color{backcolour},   
    commentstyle=\color{codegreen},
    keywordstyle=\color{magenta},
    numberstyle=\tiny\color{codegray},
    stringstyle=\color{codepurple},
    basicstyle=\ttfamily\footnotesize,
    breakatwhitespace=false,         
    breaklines=true,                 
    captionpos=b,                    
    keepspaces=true,                 
    numbers=left,                    
    numbersep=5pt,                  
    showspaces=false,                
    showstringspaces=false,
    showtabs=false,                  
    tabsize=2
}

\lstset{style=mystyle}

% Define the abs() function
\newcommand{\abs}[1]{\left\lvert #1 \right\rvert}

% Define the norm() function
\newcommand{\norm}[1]{\left\lVert #1 \right\rVert}

% Define the inner product using angle brackets
\newcommand{\inner}[2]{\left\lange #1, #2 \right\rangle}

% Define the argmin operator
\DeclareMathOperator*{\argmin}{argmin}

% Define the dist operator
\DeclareMathOperator*{\dist}{dist}

% Define the resolvent operator
\DeclareMathOperator{\res}{Res}

% Define the proximal operator
\DeclareMathOperator*{\prox}{prox}

% Redefine the proof environment to be completely silent
\renewenvironment{proof}%
{\noindent}%
{\hfill$\square$\par}

% Defaults used when private.tex is missing
\newcommand{\SupervisorOne}{Name One}
\newcommand{\SupervisorTwo}{Name Two}
\newcommand{\StudentNumber}{Student Number}

% Contains the real names, don't push private.tex on public .gitignore.
\IfFileExists{private.tex}{\input{private.tex}}{}

% Start section counter from 0
%\setcounter{section}{-1}

%%%%%%%%%%%%%%%%%%%%%%%%%%%%%%%%%%%%%%%%%%%%%%%%%%%%%%%%%%%%%%%%%%%%
%%-----------------------DOCUMENT BEGIN---------------------------%%
%%%%%%%%%%%%%%%%%%%%%%%%%%%%%%%%%%%%%%%%%%%%%%%%%%%%%%%%%%%%%%%%%%%%
\begin{document}

\begin{titlepage}
\thispagestyle{empty}
\title{
\includegraphics[width=19cm]{Extras/Mathlogo.PNG} \\
\vspace{3cm}
\begingroup
\setstretch{4}\fontsize{35}{10}\selectfont\fontdimen2\font=0.8ex

% Project Title  %
\parbox{15cm}{\center{\textBF{Project Title}}}

% Provisional Working Title
\vspace{1.5cm}
\parbox{20cm}{\centering\large\textbf{Modelling the Rainfall-Runoff Phenomenon using Shallow Water Equations}}
\endgroup}
\date{}
\maketitle

% Proposal Text & Name & Student Number
%\vspace{-1cm}
\begin{center}
	{\Large Thesis for the Master of Science in Applied Mathematics\par}
	\vspace{0.9cm}
	{\Large \textbf{Konstantinos Gkaras}\\[2mm] \StudentNumber\\[2mm]}
	\today
\end{center}

% List of supervisors
\vspace{1.6cm}
\begin{center}
	\begin{minipage}[t]{0.48\textwidth}
		\centering
		\textbf{First Supervisor}\\[0.2cm]
		{\large \SupervisorOne}\\[1.2cm]
	\end{minipage}
	\hfill
	\begin{minipage}[t]{0.48\textwidth}
		\centering
		\textbf{Second Supervisor}\\[0.2cm]
		{\large \SupervisorTwo}\\[1.2cm]
	\end{minipage}
\end{center}

\end{titlepage}
\newpage
% Define new page geometry because the default was changed for title page.
\newgeometry{top=1in,bottom=1in,right=1in,left=1in}

% Add the Abstract before the table of contents
% Abstract for the Rainfall-Runoff Project %

\begin{abstract}
	Short description of the project.
\end{abstract}


% Table of contents
\tableofcontents

% Input document sub-files, following the divide and conquer strategy.
% Introduction Section for the Rainfall-Runoff Project %

\section{Introduction}
\label{sec: intro}

Placeholder...
					% Introduction to the paper
% Models Section for the Rainfall-Runoff Project %

\section{Model}
\label{sec: model}

Placeholder...
					% Modelling Rainfall-Runoff with Moments
% Rainfall Dynamics for the Rainfall-Runoff Project %

\section{Rainfall}
\label{sec: rain}

Here comes the rain again...
				% Rainfall dynamics
% Infiltration Dynamics for the Rainfall-Runoff Project %

\section{Infiltration}
\label{sec: infiltration}

Placeholder...
					% Infiltration dynamics
% Numerical Scheme Section for the Rainfall-Runoff Project %

\section{Numerical Scheme}
\label{sec: numerics}

Placeholder...
				% Numerical Discretization Schemes
% Simulations & Test Cases Section for the Rainfall-Runoff Project %

\section{Simulations}
\label{sec: sims}

Escape the matrix??
				% Numerical Simulations

\end{document}
