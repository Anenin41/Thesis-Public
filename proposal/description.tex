% Proposal Main Body %
% Author: Konstantinos Garas
% E-mail: kgaras041@gmail.com // k.gkaras@student.rug.nl
% Created: Sun 12 Oct 2025 @ 19:10:54 +0200
% Modified: Fri 17 Oct 2025 @ 15:39:28 +0200

\section*{Description}

The goal of this project is to develop and validate an extended shallow-water model that can effectively simulate the rainfall-runoff phenomenon. First, I will introduce the baseline model, on which all additional theoretical arguments will be based. Then, new terms will be added to the equations of the model, taking into consideration both rainfall and infiltration dynamics. After the model has been created, it will be further analysed to check its stability and if it truly works as intended. 

Furthermore, because numerical software already exists for the baseline shallow-water model, in parallel with the development of the theoretical framework, I will also be reading the documentation of the code. Once the model is mathematically sound, I will extend the numerical software, by incorporating these new dynamics for rainfall at the surface of the water body, and infiltration at its depth. The overarching goal of the project is to create an easy-to-use model that is both mathematically and computationally effective in modelling the rainfall-runoff phenomenon, as well as its possible applications to real-life scenarios.

\section*{Timetable}
Below is the suggested timetable for working this project. The idea is to hand-in the final version of the paper close to the end of June 2026. As such, the starting date was chosen according to the guidelines listed on the Brightspace page, specifically the one that the proposal must be handed in at least four weeks (minimum) before the starting date.

Three and a half months later, the midterm date is conveniently defined as the 16th of March 2026. Performing this calculation once again, we obtain the end date of the project as the last Monday (29th) of June 2026, with the planned presentation taking place on the 15th of June. The date of the presentation is not set in stone, and it is subject to change as it is dependant on the availability of the supervisors.

\begin{itemize}
	\item Starting Date: Monday, December 1st 2025
	\item Mid-term Date: Monday, March 16th 2026
	\item Pre-planned Presentation Date: Monday, June 15th 2026
	\item End Date: Monday, June 29th 2026
\end{itemize}

\begin{table}[H]
	\begin{center}
		\begin{tabular}{|l|l|l|}
			\hline
			\textbf{Date}	&	\textbf{Topic} \\
			\hline
			Part 1	(Dec. - Jan.) & Literature \& Documentation Review \\
			Part 2	(Jan. - Feb.) & Model Development \\
			Part 3	(Feb. - Apr.) & Model Analysis \\
			Part 4	(Apr. - May) & Benchmarking \& Numerical Simulation \\
			Part 5	(May - June) & Revision, Review \& Finalization \\
			\hline
		\end{tabular}
	\end{center}
\end{table}

\begin{itemize}
	\item \textbf{Part 1}: Gathering and reviewing relevant literature on the subject. Developing hands-on experience with the software that I will be using later on in the project.
	\item \textbf{Part 2}: Review of the existing shallow-water model. Extending it to take into consideration rainfall-runoff dynamics.
	\item \textbf{Part 3}: Conducting stability analysis and making additional modifications to the model to ensure that it performs as expected. Implementing the model within the Python software suite.
	\item \textbf{Part 4}: Benchmarking of the rainfall-runoff model through extensive numerical simulation of both test and real-life cases.
	\item \textbf{Part 5}: Revision of the final paper. Listing possible limitations and extensions. Final review and feedback from the supervisors.
\end{itemize}
